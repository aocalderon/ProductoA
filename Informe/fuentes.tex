\section{Cobertura de acueducto}
La cobertura del servicio de acueducto corresponde al porcentaje de predios residenciales con acceso al servicio de acueducto; entendiéndose como predios 
residenciales aquellos con estratos 1, 2, 3, 4, 5, 6 y los predios aún no estratificados pero reconocidos por la alcaldía como residenciales. Esta cobertura se 
obtiene a través de la información consignada por los alcaldes en el ``Reporte de Estratificación y Coberturas'', dispuesto en el Sistema Único de Información 
del módulo alcaldes.  La información fue recopilada desde el portal de Terridata\footnote{\url{https://terridata.dnp.gov.co/}}, una 
aplicación proveida por el Departamento Nacional de Planeación de Colombia para la consulta estadística por parte de entes territoriales de de indicadores sobre 
diversos sectores y temáticas.

\section{Índice de Cobertura de Energía Eléctrica - ICEE}

Este índice es calculado por la Unidad de Planeación Minero Energética (UPME) que mide la tasa de viviendas en sector residencial que tienen infraestructura 
eléctrica disponible y es proveido por el Sistema Interconectado Nacional o mediante soluciones aisladas dentro de las zonas nacionales de interconexión (ZNI). 
 El índice fue calculado con vigencia al 2018.  El conjunto de datos fue recopilado desde la página web de la 
entidad\footnote{\url{https://tinyurl.com/5hduxcnw}}.

\section{Rendimiento promedio de cultivos}

El rendimiento de cultivos es un indicador de la cantidad de toneladas producidas por hectárea para un cultivo particular. Esta información se obtiene a partir 
de las Evaluaciones agropecuarias municipales\footnote{\url{https://www.upra.gov.co/web/guest/eva-2020}}, que consisten en la recolección de información a 
través de formularios diseñados e implementados con un aplicativo web, con módulos para los cultivos transitorios, anuales o permanentes. 

Esta operación fue desarrollada durante los meses de octubre a diciembre de 2020, recolectando información de 1099 municipios y 197 cultivos. En cuanto al 
procesamiento de datos, se obtuvo el rendimiento promedio para todos los tipos de cultivos presentes en cada municipio, priorizando el valor más reciente para 
cultivos transitorios y utilizando el valor anual para cultivos permanentes. Con respecto a los valores nulos, se utilizó el valor de -1 para aquellos 
municipios en los que no se tuviera información disponible.  Los datos puedes ser descargados desde la aplicación 
web\footnote{\url{https://tinyurl.com/yp4jmv9n}} proveida por la entidad.
 
\section{Tasa de deserción intra-anual del sector oficial}

La tasa de deserción intra-anual del sector oficial corresponde al porcentaje de estudiantes de educación básica y media (transición a once) que abandonan el 
sistema educativo antes de finalizar el año lectivo con respecto a la cantidad total de estudiantes matriculados a inicio del año. Esta información es 
recopilada por el Ministerio de Educación y disponible en Terridata, una herramienta para fortalecer la gestión pública a partir de datos a nivel municipal, 
presentando indicadores estandarizados. Con respecto a los valores nulos, se utilizó el valor de -1 para aquellos municipios en los que no se tuviera 
información disponible.

\section{Índice de Informalidad}

El índice se calcula como el porcentaje de personas desocupadas informalmente con respecto a la población total.  Este indicador también se encuentra 
disponible en Terridata y representa el porcentaje de personas que no cuentan con un trabajo formal en cada una de las entidades territoriales con respecto a 
la población total de cada municipio mismo. Esta tasa calculada con información de la FILCO (Ministerio del Trabajo) y resulta como el complemento del 
indicador ``Porcentaje de personas ocupadas formalmente con respecto a la población total'' (P. Formal) y se define de tal manera según la entidad citada: 

$$P. Informal = 1 - P. Formal$$

\section{Biomasa sobre el suelo}

El proyecto GlobBiomass \cite{santoro_globbiomass_2018, santoro_detailed_2018} de la Agencia Espacial Europea ha recopilado datos y provee conjuntos de libre 
descarga con información referente al volumen de biomasa en bosques y la métrica de biomasa por encima del suelo, conocida como AGB, para el año 2018.  El AGB 
se refiere a la masa expresada como peso seco de las partes arbóreas de toda especie vegetal viva a excepción de sus raíces.  Sus unidades se miden en toneladas 
por hectárea (Mg/ha) y la resolución espacial de este conjunto es de 1 Km$^2$.

Se descargaron dos conjuntos de datos que cubrían el área de Colombia.  Usando los polígonos correspondientes para el país se recortaron las áreas necesarias de 
cada conjunto y se unieron posteriormente para conformar un referente de AGB para todo el territorio nacional.

\section{Índice de vegetación NDVI}

El índice de diferencia normalizada de la vegetación (NDVI) es un indicador de la vitalidad de la flora presente en un área determinada.  Este se calcula a 
partir de las bandas infrarrojas y el espectro correspondiente al rojo que suministran diversos sensores de satélite.  El rango de valores oscila entre 0 y 1 
donde los valores más altos indican vegetación más saludable.  El programa Copérnico de la Unión Europea, a través del portal Copernicus Global Land 
Service\footnote{\url{https://land.copernicus.eu/global/products/ndvi}}, provee imágenes globales de NDVI para todos los meses del año después de promediar 
diferentes muestras tomadas por el satélite PROBA-V durante el periodo de 2015 al 2019.  La resolución espacial de los conjuntos de datos es de 1 Km$^2$. 

Para este estudio se tomaron una imagen para cada mes del producto Short Term Statistics Versión 3.  Las imágenes se recortaron de acuerdo a la zona de estudio 
 y se promediaron las 12 imágenes resultantes para obtener un promedio anual representativo del índice NDVI para el territorio Colombiano.  El portal para la 
descarga de imágenes y conjuntos de datos se puede acceder a través de su sitio web\footnote{\url{ https://tinyurl.com/ekzekr4d}}.

\section{Porcentaje de nubosidad}

El proyecto EarthEnv \cite{wilson_remotely_2016}  es un esfuerzo colaborativo entre diversos expertos para desarrollar un conjunto de datos de escala global con 
capas de resolución de 1 Km$^2$ para monitorear y modelar los ecosistemas, biodiversidad y clima.  El trabajo fue soportado por NCEAS, NASA, NSF y la 
Universidad de Yale.  

Para el caso del porcentaje de nubosidad se promediaron las frecuencias mensuales dentro de un periodo de 15 años (2001-16) de dos muestras diarias de los 
satélites MODIS.  Del conjunto global se extrajo el área correspondiente a Colombia con valores entre 4000 y 10000 unidades.  Estos valores deben ser divididos 
por 100 para indicar el valor porcentual del número de días con nubosidad al año.  El conjunto de datos global puede ser descargado desde la página web del 
proyecto\footnote{\url{http://www.earthenv.org/cloud.html}}.

\section{Irradiación solar}

El Banco Mundial a través de su catálogo de datos\footnote{\url{https://datacatalog.worldbank.org/search/dataset/0038645}} distribuye diferentes capas alusivas 
a variables de energía solar, entre ellas el índice de irradiación global horizontal (GHI).  Este recurso fue desarrollado por SolarGIS 
(\href{https://solargis.com}{solargis.com}) y se provee por medio del Atlas Solar Global\footnote{\url{https://globalsolaratlas.info/download/colombia}}.  Los 
valores de GHI se miden en kWh/m$^2$ con una resolución espacial nominal de 250m.  A través del portal web del atlas global se permite descargar conjuntos 
específicos para cada país por lo que solo fue necesario ajustar los límites y tamaño de píxel de este conjunto para coincidir con aquellos de los demás 
indicadores.

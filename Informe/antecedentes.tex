\section{El Atlas Solar Colombiano}
Colombia recibe abundante irradiación solar con una media de 4.5 kWh/m$^2$, por encima de la media mundial de 3.9 kWh/m$2$. Esta radiación solar media se 
mantiene casi constante durante todo el año, lo que convierte a Colombia en un lugar ideal para implementar proyectos solares fotovoltaicos (Sofia Orjuela, 
Leon; Mendoza, 2021 \textbf{[Pendiente]}) . El gobierno colombiano ha puesto en marcha una ley para estimular la implantación de sistemas fotovoltaicos a 
pequeña y gran escala mediante la concesión de incentivos fiscales (Ministerio de Minas y Energía, y Unidad de Planeación Minero Energética (UPME, 2014 
\textbf{[Pendiente]}) . Sin embargo, a pesar de los esfuerzos del gobierno por promover las instalaciones fotovoltaicas, la falta de una base de datos de 
información meteorológica sólida y de fácil acceso dificulta la realización de análisis de viabilidad de las instalaciones fotovoltaicas y la evaluación de su 
capacidad energética. 

En Colombia, el Instituto de Hidrología, Meteorología y Estudios Ambientales (IDEAM) proporciona mediciones in situ de datos meteorológicos. Sin embargo, hay 
pocas estaciones meteorológicas en el territorio colombiano y la calidad de los datos se ve afectada por la falta de mantenimiento y calibración de los 
sensores. Aunque algunos sitios web intentan proporcionar estimaciones de la producción fotovoltaica, no existe una herramienta especializada que tenga en 
cuenta las necesidades del país, los requisitos del usuario y el idioma nativo de éste. Por lo tanto, una herramienta que tenga en cuenta estos requisitos 
podría impulsar los proyectos de energía solar fotovoltaica en el país.

Para resolver este problema, se creó una herramienta web interactiva que permite visualizar datos meteorológicos y evaluar el potencial actual y futuro de 
generación fotovoltaica en todo el territorio colombiano. Esta herramienta, llamada Atlas Solar Colombiano (\url{http://162.240.212.193:3000}), se basa en dos 
bases de datos meteorológicos: una con datos históricos extraídos de la información de imágenes satelitales y otra con datos de modelos de proyección del 
cambio climático. Además, el Atlas cuenta con dos modelos de generación fotovoltaica, uno básico y uno avanzado, que permite estimar la generación de una 
instalación solar fotovoltaica, de acuerdo con las necesidades del usuario. El Atlas proporciona un mapa interactivo que muestra:

\begin{itemize}
    \item Datos históricos para el periodo comprendido entre 1998 y 2019 obtenidos del National Renewable Energy Laboratory: irradiación global horizontal, 
irradiación normal directa, irradiación difusa horizontal, ángulo cenital solar, velocidad del viento y temperatura ambiente. Esta base de datos fue validada 
con mediciones in situ proporcionadas por el IDEAM.
    \item Datos de escenarios de cambio climático a escala regional para el periodo 2070-2099 obtenidos del Coordinated Regional Downscaling Experiment: 
irradiación global horizontal, velocidad del viento y temperatura ambiente bajo dos escenarios de cambio climático.
    \item Una calculadora de generación fotovoltaica que permite a los usuarios estimar la posible potencia fotovoltaica generada por un sistema fotovoltaico 
personalizado en una ubicación específica. Un modelo básico proporciona parámetros por defecto para el sistema fotovoltaico, y todos los parámetros 
fotovoltaicos pueden personalizarse a través del modelo avanzado.
\end{itemize}

Esta herramienta interactiva permitirá a los inversores evaluar el potencial actual y futuro de generación fotovoltaica en cualquier lugar de Colombia. Se trata 
de la primera herramienta interactiva en línea que permite a los usuarios estudiar el potencial de energía fotovoltaica en Colombia a partir de una sólida base 
de datos y teniendo en cuenta las proyecciones del cambio climático.

\section{Proyecto de Desarrollo con Enfoque Territorial - PDET}
Como apoyo al fortalecimiento del Programa Colombia Sostenible con los Proyectos de Desarrollo con Enfoque Territorial -PDET el investigador Iván Carrol, de la 
Facultad de Ingeniería de la Universidad de los Andes, realizó 1260 mapas utilizando imágenes Sentinel 2 del Programa Copernicus de la Agencia Espacial Europea 
por medio de la aplicación de índices de cobertura con los cuales se creó una línea base del año 2020 que posteriormente en el año 2025 se comparará para 
determinar los efectos de la implementación de los proyectos de desarrollo en subregiones del país donde se ubican los PDET. Este proyecto fue realizado en 
conjunto con la Facultad de Economía de la Universidad de los Andes y la Universidad de Antioquia.

Adicionalmente, atendiendo el Objetivo de Desarrollo Sostenible – ODS 11 sobre ciudades y comunidades sostenibles, Carroll aplicó un índice de funcionalidad
ecológica del suelo mediante inteligencia artificial en imágenes Sentinel 2 para alertar a tomadores de decisiones, principalmente alcaldes, de 50 ciudades de 
América Latina y el Caribe sobre la baja calidad de vida, el crecimiento acelerado y la reducción en el acceso a zonas verdes de algunas urbes. Esta 
investigación está en curso de ser publicada y presentada.

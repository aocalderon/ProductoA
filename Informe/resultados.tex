
Como primer resultado de esta investigación se ha consolidado una base de datos georreferenciada de indicadores socio-económicos y variables físicas relacionadas a la identificación
de zonas de interés para la implementación de soluciones de energía alternativa usando recursos de biomasa y solar.  Las capas procesadas se recogieron en un
único archivo con formato geopackage, formato soportado por el Open Geospatial Consortium (OGC\footnote{\url{https://www.ogc.org/}}) como estandar 
interoperable, que contiene el total de los indicadores mencionados en la sección \ref{sec:fuentes}.  El archivo se distribuye con este informe como un 
anexo digital dentro de la carpeta \textit{indicadores}.

Con las capas de entrada disponibles se trabajó la metodología propuesta a travéz de la implementación de un modelo que permite automitizar los pasos 
generales descritos en la sección \ref{sec:metodologia}.  El modelo toma como entrada las capas de indicadores mencionadas anteriormente y permite la 
generación de los índices necesarios para la selección de las zonas de implementación.  Este modelo se construyó usando la herramienta Graphical Modeler de la 
aplicación de software libre QGIS 3.22.  Dicho software es de libre descarga y de acceso público desde su sitio web\footnote{\url{https://www.qgis.org/}}.  El 
modelo también se distribuye como anexo digital a este informe en formato model3, que puede ser abierto y ejecutado usando QGIS 3.22, y como código Python, que 
igualmente puede ser ejecutado desde la consola de dicha herramienta.  Los archivos mencionados se encuentran dentro de la carpeta \textit{modelo}.  La figura 
\ref{fig:modelo} muestra una captura del modelo construido y los pasos ejecutados para la generación de los indices de potencialidad.

Por último, se presentan los índices creados por la metodología propuesta y que soportaran la selección de zonas de trabajo para la puesta en marcha de las 
siguientes fases de la investigación.  El primer índice se denomina índice de potencialidad y es una capa raster que agrega las capas transformadas y 
ponderadas de los indicadores iniciales y presenta un mapa de Colombia con valores entre 0 y 1 para cada píxel de la imagen.  Los valores más altos representan 
aquellos lugares con mayor potencial para la implementación de las soluciones de biomasa y solar teniendo en cuentas indicadores socio-económicos. 
 La figura \ref{fig:indice} ilustra el índice de potencialidad en una rampa de colores tipo semáforo donde el color verde más intenso significa los lugares más 
idóneos.  Las zonas en rojo corresponden a las áreas ambientalmente protegidas del territorio Colombiano y que han sido descartadas por tal motivo.

A partir del índice de potencialidad se realizó una agregación a nivel de municipios.  En esta operación se promediaron los valores de todos los píxeles 
dentro de cada uno de los municipios y se le asignó dicho promedio a cada uno de ellos.  Esto permite generar un orden de los municipios de mayor a menor de 
acuerdo al valor promedio del índice obtenido.  Posteriormente, se filtraron los municipios PDET del resultado anterior con el objetivo de seleccionar los 
puntajes más altos dentro de esta categoría.  Las figuras \ref{fig:pormunicipios} y \ref{fig:porpdet} ilustran la agregación por municipios y el filtrado por 
municipios PDET respectivamente.  Igualmente, los índices de potencialidad se adjuntan a este informe como anexos digitales bajo la carpeta \textit{indices}.



% Conclusión: listado de municipios con alto potencial.
% 
% Resultados.
% Mostrar mapa final. Y una tabla con los primeros 10 municipios. Justificación de los pesos seleccionados con base en los objetivos del proyecto.
% 
% Conclusiones.
% Indicar cuál es la zona seleccionada que será utilizada para las siguientes fases del proyecto. Sobre está zona se hará el estudio de concepto. Describir
% brevemente lo que se hará.
% Metodología parametrizable para la priorización/selección de la zona bajo estudio.
% Posibilidad de incluir otros indicadores, tanto socio económicos como físicos, en estudios posteriores con base en los objetivos del proyecto.
% Importante la disponibilidad de datos abiertos de fuentes oficiales.

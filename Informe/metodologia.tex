Para la selección de un conjunto de ubicaciones idóneas para la implementacion de las soluciones propuestas se decidió seguir un enfoque de selección 
multicriterio soportado en operaciones de lógica difusa.  El diagrama \ref{fig:plan} ilustra esquemáticamente los pasos de la metodología propuesta.  Esta 
metodología puede dividirse en tres partes para su explicación.  Primero, un conjunto de capas son preparadas a partir de una serie de indicadores 
socio-económicos y físicos. Luego, dichas capas son transformadas para estandarizar su magnitud y dimensiones.  Finalmente, se asigna un peso a cada capa de 
acuerdo a su importancia y  se aplica una operación de agregación sobre todas las capas para consolidar un único resultado.  A continuación se explica en más 
detalle los pasos seguidos en cada parte de la metodología.

\begin{figure}[h!]
    \centering
    \includegraphics[width=1\textwidth]{figures/plan}
    \caption{Diagrama esquemático de la metodología propuesta.}
    \label{fig:plan}
\end{figure}

En la selección \ref{sec:fuentes} se establecieron las fuentes de datos de los indicadores tenidos en cuenta dentro de la metodología.  En la primera parte 
de la metodología se hizo necesario hacer cierto tratamiento en cada una de ellas.  En particular, todas las capas referentes a indicadores socio-económicos 
se obtuvieron en un esquema tabular, esto es una tabla donde aparecia el valor de cada indicador junto con otras variables relacionadas al fenómeno a nivel de 
los municipios de Colombia.  Se tuvó en cuenta el código DANE de cada municipio junto con el valor relacionado a la variable en cuestión.  
Posteriormente, se ejecutó un cruce de datos entre la información extraida y el mapa oficial DANE para georreferenciar los datos y generar una vista de la 
distribución geográfica del indicador.  De esta manera se obtienen una serie de mapas que ilustran la composición del indicador para cada municipio.

Para las variables física también se hizó necesario aplicar ciertas operaciones espaciales para ajustar los conjuntos de datos iniciales a las condiciones del 
área de estudio.  Por lo general, para estos casos, los estudios son de caracter global, por lo que se requeria extraer los datos solo para el territorio 
Colombiano.  Según el caso fue necesario aplicar operaciones de agregación espacial, si los datos para Colombia se presentaban en más de una fuente, o 
temporales, si se proveian datos a nivel mensual y lo requerido eran promedios anuales.  

Para la segunda parte de la metodología, se requiere que todas las capas se encuentren en una version raster de idénticas dimensiones (esto es 
una imagen con igual número filas y columnas donde el valor de cada pixel sea el valor del indicador que representa).  En esta etapa, se aplicaron operaciones 
espaciales a las capas de indicadores socio-económicos para convertir su version vectorial (polígonos representando cada municipio) a capas raster.  Como 
dimensiones específicas se seleccionó un área de 1666 columnas por 1972 filas que cubre la extensión del mapa oficial DANE usando un sistemas de coordenadas 
EPSG:3857 (Pseudo-Mercator).  La resolución espacial obtenida fue de aproximadamente 1 Km$^2$.  Se reprojectaron y se ajustaron todas las capas (incluyendo las 
variables físicas) a dichas dimensiones y al sistema de coordenadas mencionado para permitir su análisis en unidades de metros.

Para permitir la aplicación de una función de agregación todas las capas deben ser tratadas para transformar sus valores y rango de datos a un estandar común.  
La lógica difusa provee una serie de funciones de membresía que permiten escalar los datos de 0 a 1 donde el usuario puede especificar valores a priorizar y 
diversas clases de curvas para transformar los datos.  La membresía lineal es la más sencilla de las funciones y se corresponde a una normalización típica.  
Los valores para cada variable se escalan, de acuerdo a las caracteristicas de la variable, dandole un valor de 0 al valor con menor prioridad y un valor de 
1 al más alto, los valores intermedios se obtienen a travéz de una interpolación directa.  

En la última parte de la metodología, contamos con una serie de capas raster ponderadas y unificadas.  Cada píxel en las imágenes se corresponde uno a uno por 
lo que cualquier operación de agregación es valida.  La operación de agregación tomará cada píxel de las capas en la misma ubicación y los agregará de acuerdo 
a su función, el resultado sera el valor del píxel en dicha ubicación para el mapa final.  De nuevo, la lógica difusa provee diferentes funciones de agregación 
pero dadas las caracteristicas del estudio se decidió utilizar una sumatoria simple.  Cabe aclarar que previo a la operación de agregación se asignaron pesos 
distintivos a cada capa.  Esto es, un valor por el cual se multiplica el valor de cada píxel antes de operarlo con los demas.  La asignación de pesos a cada 
capa se hace de manera coordinada entre los interesados y expertos y puede ser facilmente editada en el modelo final para su posterior ejecución.  

El mapa obtenido en esta estapa es finalmente escalado, entre 0 y el máximo valor posible de acuerdo a los pesos asignados, obteniendo una capa raster con 
valores entre 0 y 1 para cada ubicación en el mapa a manera de un índice.  Para obtener un indicador a nivel de municipio, se utiliza una capa adicional para 
agregar por zonas y calcular el promedio de todos los píxeles contenidos en un respectivo municipio.  Esto entrega una capa vectorial, con cada municipio 
relacionado con su valor promedio de acuerdo al puntaje obtendio en el índice anterior.  Esta lista de municipios y su valor promedio se utiliza como un 
ranking para la selección de los municipios con mayor potencial para la implementación de las soluciones propuestas en las posteriores etapas de la 
investigación.

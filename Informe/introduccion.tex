En junio de 2020, entregamos al Fondo Regional para la Cooperación Triangular en América Latina y el Caribe, a través de la GIZ nuestra postulación al proyecto
de Cooperación Triangular Alemania-Chile-Colombia, con el siguiente título: ``Proyecto de Cooperación Triangular para impulsar la reactivación económica a
través de la generación de mapas interactivos del potencial energético de fuentes renovables para la planeación en Chile y Colombia''. El objetivo
fue: Desarrollar herramientas que, a partir de uso, procesamiento y análisis de imágenes satelitales, permitan el establecimiento de estrategias de seguridad
energética y alimentaria para mitigar los efectos negativos provocados por la emergencia sanitaria del COVID 19 en comunidades vulnerables y los cuales han
generado importantes retrasos en el cumplimiento de la Agenda 2030 para el Desarrollo Sostenible. Se propuso hacerlo a través de la ubicación geográfica de
estas comunidades por medio de áreas de estudio, las cuales pueden incluir información respecto del riesgo latente al virus SARS-CoV-2 por los perfiles de
movilidad interna, migración y perfiles de distanciamiento social obtenidos a partir de datos agregados de movilidad y contactos obtenidos a partir de
información de dispositivos celulares. Una vez definidas, se procede con la cuantificación del potencial de generación de energía a partir de biomasa residual y
de radiación solar, determinando también la productividad agrícola y realizando análisis para identificar el potencial de generación de energía solar a través
de techos para dichas áreas.

Por parte de Colombia, se contó con la participación en calidad de entidad ejecutora de la Universidad de los Andes, con el apoyo del Ministerio de Minas y
Energía, Ministerio de Ambiente y Desarrollo Sostenible, Ministerio de Agricultura y Desarrollo Rural, Unidad de Planeación Minero-Energética - UPME,
Instituto Geográfico Agustín Codazzi - IGAC, Instituto de Hidrología, Meteorología y Estudios Ambientales - IDEAM.

Por parte de Chile, se contó con la participación del Centro de Energía de la Universidad de Chile con el apoyo del Ministerio de Energía de Chile y Fraunhofer
Chile.

Recibimos una respuesta favorable en febrero de 2021, fecha a partir de la cual empezamos los trámites de contratación con las GIZ-Chile y GIZ-Colombia. Aunque
la propuesta se había planteado inicialmente para una ejecución de 24 meses, luego de legalizar el contrato entre GIZ-Colombia y Uniandes, el tiempo de
ejecución se pactó para 10 meses. En dicho contrato se tiene un compromiso de los siguientes productos específicos:

\begin{itemize}
\item Base de datos de las comunidades y territorios de estudio
\item Prueba de concepto de exploradores de energía de biomasa residual agrícola
\item Explorador solar en Colombia sobre la base de las experiencias y desarrollos técnicos
alcanzados en Chile
\item Sistema que integra los exploradores de energía solar y de biomasa residual agrícola
desarrollados para Colombia y Chile
\item Prueba de concepto del aprovechamiento de energía solar en techos solares fotovoltaicos
\item Estimación de indicadores de impacto de los desarrollos tecnológicos en las áreas de estudio
seleccionadas
\item Estrategia para la continuación, expansión y transferencia de tecnología
\end{itemize}


Este informe se realiza en cumplimiento a los compromisos contractuales y corresponde a los resultados del producto A: Base de datos de las comunidades y
territorios de estudio. Se incluye una sección de Antecedentes que contempla:
- La explicación de un explorador solar previo (Sección 2.1)
- El Desarrollo de mapas para zonas en Colombia con Enfoque Territorial (PDET) (Sección 2.2)
Estos desarrollos no hacen parte del presente proyecto, pues se obtuvieron en fechas posteriores a nuestra aplicación al Fondo Regional para la Cooperación
Triangular en América Latina y el Caribe, pero antes de la iniciación de este contrato. Se mencionan en este informe para hacer claridad al respecto y dejar
estos desarrollos fuera de los que se obtengan en la ejecución del presente contrato.

Por otra parte, con base en información disponible y considerando que los limitados tiempos de ejecución, se hizo una selección de los indicadores que
permitirían la evaluación cuantitativa de las zonas en Colombia sobre las que podrían realizarse los desarrollos. Esta selección se puso a consideración de
representantes de todas las entidades interesadas (mencionadas antes) y los resultados consolidados de índices seleccionados se listan y explican brevemente en
la Sección 3 de este documento.

La metodología empleada para el cálculo de las zonas y los resultados y conclusiones se muestran en la Sección 4.

La zona seleccionada para este estudio corresponde a 5 municipios del departamento de Putumayo el cual está ubicado al suroeste del país, en la región
Amazónica, los detalles se incluyen en la Sección 4. El departamento de Putumayo tuvo altos puntajes en todos los indicadores. Es importante anotar que la
mayoría de los municipios del Putumayo hace parte de los territorios focalizados PDET.
